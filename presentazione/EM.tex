
\documentclass[hyperref={pdfpagelabels=false},slidestop,mathserif,red]{beamer}


\mode<presentation>{
  \usetheme{Copenhagen}

  \setbeamercovered{transparent}
  % or whatever (possibly just delete it)
}
%\usecolortheme{lily}
\usecolortheme{orchid}

\usepackage[italian]{babel}
% or whatever

\usepackage[latin1]{inputenc}
% or whatever

\usepackage{times}
%\usepackage{cite}
\usepackage[T1]{fontenc}
%\usepackage{fourier}
\usepackage{eulervm}
\usepackage{algorithm}
% Or whatever. Note that the encoding and the font should match. If T1
% does not look nice, try deleting the line with the fontenc.
\usepackage{pgf}

\title[Elaborato di Apprendimento Automatico] % (optional, use only with long paper titles)
{}

\subtitle
{Accelerometer Based Gesture Recognition using HMMs} % (optional)

\author[Andrea~Tarocchi, Marco~Magnatti] % (optional, use only with lots of authors)
{Andrea~Tarocchi, Marco~Magnatti}
% - Use the \inst{?} command only if the authors have different
%   affiliation.


\date[Exam] % (optional)
{8 gennaio 2009}

\subject{Talks}
% This is only inserted into the PDF information catalog. Can be left
% out

\begin{document}

\section{Algoritmo EM}

\begin{frame}{L'algoritmo EM}{}
  % - A title should summarize the slide in an understandable fashion
  %   for anyone how does not follow everything on the slide itself.

\begin{block}{}
L'algoritmo EM viene utilizzato in situazioni in cui si vuole stimare un set di parametri $\theta$ di una distribuzione di probabilit\`a potendo osservare solo una porzione dei dati prodotti da essa.
\end{block}

\begin{block}{}
 Usa la stima corrente dei parametri da determinare per calcolare il valore atteso dei dati nascosti.
\end{block}

\begin{block}{}
 Dopodich\'e, quanto appena calcolato viene usato per migliorare l'ipotesi corrente sui parametri.
\end{block}

\begin{block}{}
 Iterando pi\`u volte, l'algoritmo aumenta la \textit{likelihood} dei dati rispetto all'ipotesi, fino a convergere a un massimo locale.
\end{block}

\end{frame}

\begin{frame}{}

\begin{block}{}
 Siano
  \begin{itemize}
   \item $X = {x_{1}, \ldots, x_{m}}$ i dati osservabili
   \item $Z = {z_{1}, \ldots, z_{m}}$ i dati nascosti
   \item $Y = X \cup Z$ i dati completi
   \item $h$ l'ipotesi corrente sui parametri $\theta$
   \item $h'$ la nuova ipotesi (migliore) da determinare
  \end{itemize}
\end{block}

\begin{block}{}
 $Z$ e (di conseguenza) $Y$ possono essere trattate come variabili aleatorie.
\end{block}

\begin{block}{}
 L'algoritmo usa l'ipotesi corrente $h$ al posto dei parametri $\theta$ per stimare la distribuzione di $Y$, che non \`e nota.\\
 Viene poi cercata l'ipotesi $h'$ a massima verosomiglianza massimizzando la funzione $E[\ln P(Y | h'))]$.
\end{block}

\end{frame}

\begin{frame}{}

\begin{block}{}
 Introduciamo
 \begin{center}
  $Q(h'|h) = E[\ln P(Y | h'))\ |\ h, X]$
 \end{center}
 che esprime $E[\ln P(Y | h'))$ come funzione della variabile $h'$, sotto l'assunzione che $\theta = h$, e data la porzione osservata $X$ dei dati.
\end{block}

\begin{block}{Algoritmo EM}
 \begin{enumerate}
  \item E-step: calcola $Q(h'|h)$, usando l'ipotesi corrente e i dati osservabili $X$ per stimare la distribuzione di $Y$.
  \item M-step: sostituisci $h$ con l'ipotesi $h'$ che massimizza $Q$:
	\begin{center}
	 $h \leftarrow \arg\max_{h'} Q(h'|h)$
	\end{center}
 \end{enumerate}

\end{block}

\begin{block}{}
 L'algoritmo converge a un massimo locale della funzione di \textit{likelihood} $P(Y|h')$.
\end{block}

\end{frame}


\begin{frame}
	\frametitle{Bibliografia}
	
		\begin{thebibliography}{}
		%
			\footnotesize
				
			\bibitem<1->[Mitchell, 1997]{Mitchell1997}
			 Tom M. Mitchell
			\newblock {\em Machine Learning}, pp 191-195.
			\newblock McGraw-Hill, 1997
		\end{thebibliography}
\end{frame}

\end{document}


